%
% Template Laporan Skripsi/Thesis
%
% @author  Rifa Faruqi
% @version 1.00
%
% Dokumen ini dibuat berdasarkan standar penulisan skripsi Fakultas MIPA, Universitas Syiah Kuala (USK).
% Template ini dimodifikasi dari versi asli yang dibuat oleh Andreas Febrian dan Lia Sadita,
% yang awalnya didasarkan pada standar IEEE dan konfigurasi LaTeX yang digunakan Fahrurrozi Rahman
% untuk laporan skripsi di Universitas Indonesia (UI).
%
% Modifikasi ini disesuaikan dengan panduan terbaru dari Fakultas MIPA, USK.
%
%
% Tipe dokumen adalah report dengan satu kolom. 
%
\documentclass[12pt, a4paper, onecolumn, oneside, final]{report}

% Load konfigurasi LaTeX untuk tipe laporan thesis
\usepackage{_internals/uithesis}


% Daftar pemenggalan suku kata dan istilah dalam LaTeX
%
% Hyphenation untuk Indonesia 
%
% @author  Andreas Febrian
% @version 1.00
% 
% Tambahkan cara pemenggalan kata-kata yang salah dipenggal secara otomatis 
% oleh LaTeX. Jika kata tersebut dapat dipenggal dengan benar, maka tidak 
% perlu ditambahkan dalam berkas ini. Tanda pemenggalan kata menggunakan 
% tanda '-'; contoh:
% menarik
%   --> pemenggalan: me-na-rik
%

\hyphenation{
    % alphabhet A
    a-na-li-sa a-tur 
    a-pli-ka-si 
    % alphabhet B
    ba-ngun-an 
    be-be-ra-pa 
    ber-ge-rak
    ber-ke-lan-jut-an 
    ber-pe-nga-ruh 
    % alphabhet C
    ca-ri copy-writing chat-gpt
    % alphabhet D
    di-sim-pan di-pim-pin de-ngan da-e-rah di-ba-ngun da-pat di-nya-ta-kan 
    di-sim-bol-kan di-pi-lih di-li-hat de-fi-ni-si di-se-su-ai-kan di-ge-ne-ra-si pre-sen-ta-si
    di-ha-rap-kan
    % alphabhet E
    e-ner-gi eks-klu-sif
    % alphabhet F
    fa-si-li-tas
    % alphabhet G
    ga-bung-an ge-rak
    % alphabhet H
    ha-lang-an
    % alphabhet I
    % alphabhet J
    % alphabhet K
    ke-hi-lang-an
    ku-ning 
    kua-li-tas ka-me-ra ke-mung-kin-an ke-se-pa-ham-an
    ke-sempurna-an ke-bu-tu-han
    % alphabhet L
    ling-kung-an
    % alphabhet M
    me-neng-ah
    meng-a-tas-i me-mung-kin-kan me-nge-na-i me-ngi-rim-kan 
    meng-u-bah meng-a-dap-ta-si me-nya-ta-kan mo-di-fi-ka-si
    meng-a-tur meng-ucap-kan ma-nual me-makan meng-or-ga-ni-sa-si-kan mark-down
    mem-ba-ngun
    % alphabhet N
    nya-ta non-eks-klu-sif
    % alphabhet O
    % alphabhet P
	pe-nye-rap-an 
	pe-ngon-trol
    pe-mo-del-an
    pe-ran  pe-ran-an-nya
    pem-ba-ngun-an pre-si-den pe-me-rin-tah prio-ri-tas peng-am-bil-an 
    peng-ga-bung-an pe-nga-was-an pe-ngem-bang-an 
    pe-nga-ruh pa-ra-lel-is-me per-hi-tung-an per-ma-sa-lah-an 
    pen-ca-ri-an peng-struk-tur-an
    per-kembang-an pe-nge-tahu-an 
    po-pu-ler per-ca-kap-an pem-buat-an
    % alphabhet Q
    % alphabhet R
    ran-cang-an res-pons
    % alphabhet S
    si-mu-la-si sa-ngat se-ki-tar
    % alphabhet T
    te-ngah
    ter-da-pat
    % alphabhet U
    % alphabhet V
    % alphabhet W
    % alphabhet X
    % alphabhet Y
    % alphabhet Z
    % special
}

% Load konfigurasi khusus untuk laporan yang sedang dibuat
%-----------------------------------------------------------------------------%
% Informasi Mengenai Dokumen
%-----------------------------------------------------------------------------%
% --------------------
% Informasi Umum
%  -------------------
% Tulis nama penulis 
\var{\penulis}{Rifa Faruqi}
% Tulis kembali nama penulis, kali ini akan diubah menjadi huruf kapital
\Var{\Penulis}{Rifa Faruqi}
\var{\tempatTglLahir}{Banda Aceh/ 29 Maret 2002}
% Tulis NPM penulis
\var{\npm}{21XX}
% Judul laporan. 
% Tuliskan tahun publikasi laporan
\Var{\bulan}{Sep}
\Var{\tahun}{2024}
% Tulis judul 
\var{\judul}{Judul/Topik}
% Tulis kembali judul laporan, kali ini akan diubah menjadi huruf kapital
\Var{\Judul}{Judul/Topik}
% Tulis kembali judul laporan namun dengan bahasa Ingris
\var{\judulInggris}{Judul/Topik}
\var{\JudulInggris}{Judul/Topik}
% Tipe laporan, dapat berisi Skripsi, Tugas Akhir, Thesis, atau Disertasi
\var{\type}{Proposal Penelitian/Tugas Akhir}
\Var{\Type}{Proposal Penelitian/Tugas Akhir}
% 
\var{\jurusan}{Jurusan ?}
% Tulis kembali tipe laporan, kali ini akan diubah menjadi huruf kapital
\Var{\Jurusan}{Jurusan ?}
% Tuliskan gelar yang akan diperoleh dengan menyerahkan laporan ini
\var{\gelar}{Sarjana Ilmu Komputer}
% Tuliskan Fakultas dimana penulis berada
\var{\fakultas}{Matematika dan Ilmu Pengetahuan Alam}
\Var{\Fakultas}{Matematika dan Ilmu Pengetahuan Alam}
% Tuliskan Program Studi yang diambil penulis
\var{\program}{Informatika Jurusan Informatika}
\Var{\Program}{Informatika Jurusan Informatika}
% 
% --------------------------------------------------------
% Informasi untuk halaman pengesahan dan surat pernyataan
% --------------------------------------------------------
% Tuliskan pembimbing 
\var{\pembimbingSatu}{Nama Pembimbing Satu}
\var{\nipPembimbingSatu}{NIP Pemb Satu}
\var{\pembimbingDua}{Nama Pembimbing Dua}
\var{\nipPembimbingDua}{NIP Pemb Dua}
% Infomasi Kaprodi, Dekan dan Koordinator TA
\var{\kaprodi}{Nama Kaprodi}
\var{\kaprodinip}{NIP Kaprodi}
\var{\dekan}{Nama Dekan}
\var{\dekannip}{NIP Dekan}
\var{\koordinatorTA}{Nama Koor TA}
\var{\koordinatorTAnip}{NIP Koor TA}
% Tuliskan tanggal pengesahan
\var{\tanggalPengesahan}{Hari, XX Bulan YYYY} 
% Tanggal pernyataan bebas plagiasi
\var{\tanggalPernyataanBebasPlagiasi}{Banda Aceh, XX Bulan YYYY}
\var{\tanggalSuratPernyataan}{Darussalam, XX Bulan YYYY}

% Tuliskan tanggal keputusan sidang dikeluarkan dan penulis dinyatakan 
% lulus/tidak lulus
\var{\tanggalLulus}{XX Juli 2019}
% 
\var{\tanggalKataPengantar}{Banda Aceh, XX Bulan YYYY}
% 
% Alias untuk memudahkan alur penulisan paa saat menulis laporan
\var{\saya}{Penulis}

%-----------------------------------------------------------------------------%
% Judul Setiap Bab
%-----------------------------------------------------------------------------%
% 
% Berikut ada judul-judul setiap bab. 
% Silahkan diubah sesuai dengan kebutuhan. 
% 
\Var{\kataPengantar}{Kata Pengantar}
\Var{\babSatu}{Pendahuluan}
\Var{\babDua}{Tinjauan Kepustakaan}
\Var{\babTiga}{Metodologi Penelitian}
\Var{\babEmpat}{Hasil dan Pembahasan}
\Var{\babLima}{Kesimpulan dan Saran}

\Var{\babEnam}{Bab Enam}
\Var{\kesimpulan}{Kesimpulan dan Saran}


% Tipe laporan, dapat berisi Skripsi, Tugas Akhir, Thesis, atau Disertasi
\renewcommand{\type}{Tugas Akhir}
% Tulis kembali tipe laporan, kali ini akan diubah menjadi huruf kapital
\renewcommand{\Type}{TUGAS AKHIR}

% Daftar istilah yang mungkin perlu ditandai 
\input{istilah}

\pagestyle{fancy}

% Awal bagian penulisan laporan
\begin{document}
%

%
% Gunakan penomeran romawi
\pagenumbering{roman}
\fancyhf{} % Bersihkan header dan footer terlebih dahulu
\fancyfoot[C]{\thepage} % Nomor halaman di tengah bawah (bisa ganti posisi jika diperlukan)


%
% load halaman judul dalam
\addChapter{Halaman Judul}
%
% Halaman Judul Laporan 
%
% @author  unknown
% @version 1.01
% @edit by Andreas Febrian
%

\begin{titlepage}

    \begin{center}
        % judul thesis harus dalam 14pt Times New Roman
        \vspace*{4em}
        {\fontsize{20}{20}
            \textbf{\Judul} \\[3em]
        }
        % Mengatur ukuran font menjadi 16pt dan menambahkan spasi vertikal
        {\fontsize{16}{20}
            \textbf{\Type} \\[3em] 
        }
        % keterangan prasyarat
        {\fontsize{12}{20}
            {Diajukan untuk melengkapi tugas-tugas dan memenuhi syarat-syarat guna memperoleh gelar 
            \gelar} \\[3em]
        }
        % Oleh
        {\fontsize{14}{20}
            {Oleh:} \\[3em]
        }
        % penulis dan npm
        {\fontsize{14}{20}
            \underline{\bo{\Penulis}} \\
            \underline{\bo{\npm}} \\[3em]
        }
        
        \begin{figure}
            \begin{center}
                \includegraphics[width=4cm]{_internals/usk_logo.png}
            \end{center}
        \end{figure}    
        \vspace*{3em}
        

        % informasi mengenai fakultas dan program studi
        \bo{
            PROGRAM STUDI \Program\ FAKULTAS \Fakultas\\ UNIVERSITAS SYIAH KUALA, BANDA ACEH \\
            \bulan, \tahun
        }
    \end{center}
\end{titlepage}

% setelah bagian ini, halaman dihitung sebagai halaman ke 2
\setcounter{page}{2}
%

% -- Halaman Pengesahan untuk sidang
\addChapter{Halaman Pengesahan}
%------------------------------------------------------------
%Approval Page
%------------------------------------------------------------
\chapter*{PENGESAHAN}


\begin{center}
    \begin{doublespace}
        \vspace{0.5cm}
        \fontsize{12pt}{10pt}\selectfont\MakeUppercase{\large{\bfseries\judul}}\par\nobreak

        \vspace{1cm}
        \fontsize{12pt}{10pt}\selectfont\MakeUppercase{\large{\bfseries\judulInggris}}\par\nobreak

        \vspace{0.9cm}
        Oleh:
        %\vspace{-0.5cm}
        \begin{singlespace}
            \begin{compactitem}
                \addtolength{\itemindent}{3cm}
                \setlength{\parsep}{0pt}
                \item[]{\makebox[2cm]{Nama\hfill} : \penulis}
                \item[]{\makebox[2cm]{NPM\hfill} : \npm}
                \item[]{\makebox[2cm]{Jurusan\hfill} : \jurusan}
            \end{compactitem}
        \end{singlespace}
        %\vspace{2.0cm} edited 16/04/2022
        \vspace{0.9cm}
        
        %Telah disetujui dan disahkan\\
        %\vspace{1.0cm}

        Menyetujui:\\
        \onehalfspacing
            \begin{tabular}{lll}
                Pembimbing I & \hspace{2.5cm} & Pembimbing II \\
                \vspace{0.3cm} & \vspace{0.3cm} & \vspace{0.3cm}\\
                \underline{\pembimbingSatu}& &
                \underline{\pembimbingDua} \\
                NIP. \nipPembimbingSatu & &  NIP. \nipPembimbingDua
            \end{tabular}	
        %\vspace{2.0cm} edited 16/04/2022
        \vspace{1.0cm}
        
        % Mengetahui dekan dan kajur informatika
        % Tanpa mengetahui dekan
        Mengetahui:\\
            \begin{tabular}{lll}
                Dekan FMIPA & &
                Koordinator Prodi S1 Informatika FMIPA \\ 
                \vspace{-0.27cm} & \vspace{-0.27cm} & \vspace{-0.27cm}\\
                Universitas Syiah Kuala, & & Universitas Syiah Kuala,\\
                \vspace{0.3cm} & \vspace{0.3cm} & \vspace{0.3cm}\\
                \underline{\dekan} & & \underline{\kaprodi}\\
                    NIP. \dekannip & & NIP. \kaprodinip
            \end{tabular}
    \end{doublespace}

    \vspace{0.9cm}
    Lulus Sidang Sarjana pada hari \tanggalPengesahan
\end{center}

\thispagestyle{empty}   % Mematikan penomoran halaman
\clearpage
% -- Batas Halaman Pengesahan

% --- Surat pernyataan bebas plagiasi 
\addChapter{Pernyataan Bebas Plagiasi}
\chapter*{PERNYATAAN BEBAS PLAGIASI}


\noindent
Saya yang bertanda tangan di bawah ini,

\vspace{-0.1cm}

\begin{table}[H]
\begin{tabular}{ll}
    \textbf{Nama lengkap}          &: \penulis \\
    \textbf{Tempat/tanggal lahir}   &: \tempatTglLahir \\
    \textbf{NPM}                   &: \npm    \\
    \textbf{Program Studi}          &: \jurusan \\
    \textbf{Fakultas}               &: \fakultas \\
    \textbf{Judul Tugas Akhir}      &: 
    \parbox[t]{0.67\textwidth}{
        \Judul
    }
\end{tabular}
\end{table}

\vspace{0.2cm}
\noindent
Menyatakan dengan sesungguhnya bahwa Laporan Tugas Akhir saya dengan judul di atas adalah \textbf{hasil karya saya sendiri} bersama dosen pembimbing dan \textbf{bebas plagiasi}.

\vspace{1cm}
\noindent
Jika ternyata di kemudian hari terbukti bahwa Laporan Tugas Akhir merupakan hasil plagiasi, saya bersedia menerima sanksi yang berlaku di Universitas Syiah Kuala.

\vspace{1cm}


\begin{tabular}{p{7.5cm}l}
	&\tanggalPernyataanBebasPlagiasi\\
	&\\
	&\\
	&\multirow{1.5}{7.5cm}{\underline{\penulis}} \\ 
	&NPM. \npm \\
\end{tabular}
\thispagestyle{empty}   % Mematikan penomoran halaman
\clearpage
% -- Batas Surat Pernyataan Bebas Plagiasi

% --- Surat pernyataan
\addChapter{Surat Pernyataan}
\chapter*{SURAT PERNYATAAN}

\begin{singlespace}

\noindent
Yang bertanda tangan di bawah ini,
\vspace{-0.1cm}

\begin{table}[H]
\centering
\begin{tabularx}{\textwidth}{llX} % X untuk kolom yang dapat menyesuaikan
1. &   Nama  		&: \penulis \\
	&   NPM       	&: \npm   \\
	&   Jurusan/Prodi &: \jurusan \\
	&   Status      	&: Mahasiswa \\  
2.	& Nama  		&: \pembimbingSatu \\
	&   NIP       	&: \nipPembimbingSatu   \\
	&   Jurusan/Prodi &: Informatika \\
	&   Status      	&: Pembimbing I \\  
3.	& Nama  		&: \pembimbingDua \\
	&   NIP       	&: \nipPembimbingDua   \\
	&   Jurusan/Prodi &: Informatika \\
	&   Status      	&: Pembimbing II   
\end{tabularx}
\end{table}

\vspace{-0.4cm}
\noindent
Dengan ini menyatakan hasil penelitian Tugas Akhir yang berjudul \textbf{\Judul} tidak dipublikasikan secara \textit{full-text} di sistem ETD (\textit{Electronic Theses and Dissertations}) Universitas Syiah Kuala hingga batas waktu 5 tahun dari tanggal kelulusan.

\vspace{0.4cm}
\noindent
Demikian surat pernyataan ini dibuat dengan sebenarnya untuk dapat dipergunakan seperlunya.

\vspace{0.4cm}
{\renewcommand{\arraystretch}{0.8}
\centering
\begin{tabularx}{\textwidth}{llX} % X untuk kolom yang dapat menyesuaikan
	&Darussalam, 14 Mei 2022		& \\
	&Yang membuat pernyataan,			& \\
	&&\\
	Pembimbing I,							&Pembimbing II,							&Mahasiswa,\\
	&&\\
	&&\\
	&&\\
	\underline{\pembimbingSatu}	&\underline{\pembimbingDua} &\underline{\penulis}\\
	NIP. \nipPembimbingSatu				&NIP. \nipPembimbingDua				&NPM. \npm \\
	&&\\
	&Mengetahui:\\			&
\end{tabularx}
}
{\renewcommand{\arraystretch}{0.8}
\begin{tabularx}{\textwidth}{llX} % X untuk kolom yang dapat menyesuaikan
	Koordinator Program Studi Informatika	&\qquad\qquad  &Koordinator TA,\\
	Universitas Syiah Kuala,&\quad\quad  &\\
	&&\\
	&&\\
	&&\\
	\underline{\kaprodi}	&\quad\quad  &\underline{\koordinatorTA}\\
	NIP. \kaprodinip						&\quad\quad  &NIP. \koordinatorTAnip				
\end{tabularx}
}
\end{singlespace}
\thispagestyle{empty}   % Mematikan penomoran halaman
\clearpage
% -- Batas Surat Pernyataan

% --- Bagian Abstrak
\addChapter{Abstrak/Abstract}
\input{src/00-front_matter/abstrak}
%
% Halaman Abstract
%
% @author  Andreas Febrian
% @version 1.00
%

\begin{singlespace} % Atur 1 spasi untuk bagian abstract sesuai pedoman
\chapter*{\textit{Abstract}}


{
	\textit{The focus of this study is the freshman student of Faculty of Psychology at University of
	Indonesia experience of acquiring, evaluating and using information, when they enroll in
	“Program Dasar Pendidikan Tinggi (PDPT)”. The purpose of this study is to understand
	how freshman students acquire, evaluate and use information. Knowing this will allow
	library to identify changes should be made to improve user education program at
	University of Indonesia. This research is qualitative descriptive interpretive. The data
	were collected by means of deep interview. The researcher suggests that library should
	improve the user education program and provide facilities which can help students to be
	information literate.}

	\vspace{1.5em}

	\noindent \textit{\textbf{Keywords:} Information literacy, information skills, information}
}
\end{singlespace}
\newpage

% --- Bagian Kata Pengantar
\addChapter{Kata Pengantar}
\input{src/00-front_matter/kata_pengantar}
%
%
%
% Daftar isi, gambar, dan tabel
%
% Atur jarak antar chapter sama seperti jarak antar section
\disableboldchapterintoc
\phantomsection
\begin{singlespace}
\tableofcontents
\clearpage

\phantomsection
\listoftables
\clearpage

\phantomsection
\listoffigures
\clearpage

\phantomsection
\listofappendices
\addcontentsline{toc}{chapter}{Daftar Lampiran}
\clearpage

\enableboldchapterintoc

\end{singlespace}

%
% Gunakan penomeran Arab (1, 2, 3, ...) setelah bagian ini.
%
\pagenumbering{arabic} % Mengatur penomoran halaman ke format Arab
\fancyhf{} % Bersihkan header dan footer
\fancyfoot[R]{\thepage} % Nomor halaman di kanan bawah
%
%
%
%-----------------------------------------------------------------------------%
\chapter{\babSatu}
\thispagestyle{fancy}
%-----------------------------------------------------------------------------%
\todo{tambahkan kata-kata pengantar bab 1 disini jika perlu}

%-----------------------------------------------------------------------------%
\section{Latar Belakang}
%-----------------------------------------------------------------------------%
\todo{tuliskan latar belakang penelitian disini}


%-----------------------------------------------------------------------------%
\section{Rumusan Masalah}
%-----------------------------------------------------------------------------%
\todo{tuliskan rumusan masalah disini, bisa berbentuk point-point}


%-----------------------------------------------------------------------------%
\section{Tujuan Penelitian}
%-----------------------------------------------------------------------------%
\todo{tuliskan tujuan penelitian disini, diangkat dari rumusan masalah, bisa berbentuk point-point}

%-----------------------------------------------------------------------------%
\section{Manfaat Penelitian}
%-----------------------------------------------------------------------------%
\todo{Tuliskan tujuan penelitian.}

% ==============HAPUS DIBAWAH INI JIKA SUDAH TIDAK PERLU===================== %
% \subsection{Batasan Permasalahan}
% %-----------------------------------------------------------------------------%
% \todo{Umumnya ada asumsi atau batasan yang digunakan untuk 
% 	menjawab pertanyaan-pertanyaan penelitian diatas.}


% %-----------------------------------------------------------------------------%
% \section{Posisi Penelitian}
% %-----------------------------------------------------------------------------%
% \todo{Posisi penelitian Anda jika dilihat secara bersamaan dengan 
% 	peneliti-peneliti lainnya. Akan lebih baik lagi jika ikut menyertakan 
% 	diagram yang menjelaskan hubungan dan keterkaitan antar 
% 	penelitian-penelitian sebelumnya}


% %-----------------------------------------------------------------------------%
% \section{Metodologi Penelitian}
% %-----------------------------------------------------------------------------%
% \todo{Tuliskan metodologi penelitian yang digunakan.}


% %-----------------------------------------------------------------------------%
% \section{Sistematika Penulisan}
% %-----------------------------------------------------------------------------%
% Sistematika penulisan laporan adalah sebagai berikut:
% \begin{itemize}
% 	\item Bab 1 \babSatu \\
% 	\item Bab 2 \babDua \\
% 	\item Bab 3 \babTiga \\
% 	\item Bab 4 \babEmpat \\
% 	\item Bab 5 \babLima \\
% 	\item Bab 6 \babEnam \\
% 	\item Bab 7 \kesimpulan \\
% \end{itemize}

% \todo{Tambahkan penjelasan singkat mengenai isi masing-masing bab.}


\input{src/01-body/02-bab2}
%-----------------------------------------------------------------------------%
\chapter{\babTiga}
\thispagestyle{fancy}
%-----------------------------------------------------------------------------%
\todo{tambahkan kata-kata pengantar bab 1 disini}

%-----------------------------------------------------------------------------%
\section{Waktu dan Lokasi Penelitian}
%-----------------------------------------------------------------------------%
\todo{tuliskan waktu dan lokasi penelitian disini}

%-----------------------------------------------------------------------------%
\section{Alat dan Bahan}
%-----------------------------------------------------------------------------%
\todo{tuliskan alat dan bahan penelitian disini}

%-----------------------------------------------------------------------------%
\section{Metode Penelitian}
%-----------------------------------------------------------------------------%
\todo{tuliskan metode penelitian disini}

\input{src/01-body/04-bab4}
\input{src/01-body/05-bab5}

% Agar nomor halaman daftar pustaka halaman pertama berada di kanan bawah karena \printbibliography menggunakan plain
\fancypagestyle{plain}{ %This is to change the \chapter functionality so the page number is in the bottom right corner.
  \fancyhf{}
  \fancyfoot[R]{\thepage}  % Nomor halaman di kanan bawah
  \renewcommand{\headrulewidth}{0pt}
  \renewcommand{\footrulewidth}{0pt}
}
% Alternatif manajemen daftar pustaka dengan \bibliography
\begin{singlespace}
    \clearpage
    \addChapter{\bo{DAFTAR KEPUSTAKAAN}}
    \printbibliography[title={Daftar Kepustakaan}]
\end{singlespace}


%
% Lampiran 
%
\begin{appendix}
	% \input{_internals/markLampiran} % Penggunaan ini hanya untuk menandai halaman LAMPIRAN, jika perlu
	% \setcounter{page}{2}\textbf{} % Reset penomoran halaman jika perlu
	%-----------------------------------------------------------------------------%
\addChapter{\textbf{LAMPIRAN}}
\chapter*{Lampiran}
\thispagestyle{fancy}
%-----------------------------------------------------------------------------%

\section{Daftar Isi Kode Program}
\end{appendix}

\end{document}