%
% Template Laporan Skripsi/Thesis
%
% @author  Rifa Faruqi
% @version 1.00
%
% Dokumen ini dibuat berdasarkan standar penulisan skripsi Fakultas MIPA, Universitas Syiah Kuala (USK).
% Template ini dimodifikasi dari versi asli yang dibuat oleh Andreas Febrian dan Lia Sadita,
% yang awalnya didasarkan pada standar IEEE dan konfigurasi LaTeX yang digunakan Fahrurrozi Rahman
% untuk laporan skripsi di Universitas Indonesia (UI).
%
% Modifikasi ini disesuaikan dengan panduan terbaru dari Fakultas MIPA, USK.
%
%
% Tipe dokumen adalah report dengan satu kolom. 
%
\documentclass[12pt, a4paper, onecolumn, oneside, final]{report}

% Load konfigurasi LaTeX untuk tipe laporan thesis
\usepackage{_internals/uithesis}


% Daftar pemenggalan suku kata dan istilah dalam LaTeX
\input{_internals/hype.indonesia}

% Load konfigurasi khusus untuk laporan yang sedang dibuat
\input{settings}

% Tipe laporan, dapat berisi Skripsi, Tugas Akhir, Thesis, atau Disertasi
\renewcommand{\type}{Proposal Penelitian}
% Tulis kembali tipe laporan, kali ini akan diubah menjadi huruf kapital
\renewcommand{\Type}{Proposal Penelitian}

% Daftar istilah yang mungkin perlu ditandai 
\input{istilah}

\pagestyle{fancy}

% Awal bagian penulisan laporan
\begin{document}
%
%
% Gunakan penomeran romawi
\pagenumbering{roman}
\fancyhf{} % Bersihkan header dan footer terlebih dahulu
\fancyfoot[C]{\thepage} % Nomor halaman di tengah bawah (bisa ganti posisi jika diperlukan)


%
% load halaman judul dalam
\addChapter{Halaman Judul}
%
% Halaman Judul Laporan 
%
% @author  unknown
% @version 1.01
% @edit by Andreas Febrian
%

\begin{titlepage}

    \begin{center}
        % judul thesis harus dalam 14pt Times New Roman
        \vspace*{3em}
        {\fontsize{20}{20}
            \textbf{\Judul} \\[3em]
        }
        % Mengatur ukuran font menjadi 16pt dan menambahkan spasi vertikal
        {\fontsize{16}{20}
            \textbf{\Type} \\[3em] 
        }
        % keterangan prasyarat
        {\fontsize{12}{20}
            {Diajukan untuk melengkapi tugas-tugas \\dan memenuhi syarat-syarat guna memperoleh gelar 
            \gelar} \\[3em]
        }
        % Oleh
        {\fontsize{14}{20}
            {Oleh:} \\[3em]
        }
        % penulis dan npm
        {\fontsize{14}{20}
            \underline{\bo{\Penulis}} \\
            \bo{\npm} \\[3em]
        }
        
        \begin{figure}
            \begin{center}
                \includegraphics[width=4cm]{_internals/usk_logo.png}
            \end{center}
        \end{figure}    
        \vspace*{2em}
        
        % informasi mengenai fakultas dan program studi
        \begin{center}
            {\fontsize{13}{20}
                \bo{PROGRAM STUDI \Program} \\
                \vspace{0.1em}
                \bo{FAKULTAS \Fakultas} \\
                \vspace{0.1em}
                \bo{UNIVERSITAS SYIAH KUALA, BANDA ACEH} \\
                \vspace{0.1em}
                \bo{\bulan, \tahun}
            }
        \end{center}
    \end{center}
\end{titlepage}

% setelah bagian ini, halaman dihitung sebagai halaman ke 2
\setcounter{page}{2}
%
% Untuk seminar proposal, uncomment bagian ini
\addChapter{Halaman Pengesahan}
\input{src/00-front_matter/halaman_pengesahan_proposal}
\thispagestyle{empty}   % Mematikan penomoran halaman
\clearpage
% -- Batas Halaman Pengesahan

% --- Bagian Kata Pengantar
\addChapter{Kata Pengantar}
\input{src/00-front_matter/kata_pengantar}
%
%
%
% Daftar isi, gambar, dan tabel
%
% Atur jarak antar chapter sama seperti jarak antar section
\disableboldchapterintoc
\phantomsection
\begin{singlespace}
\tableofcontents
\clearpage

\phantomsection
\listoftables   % Daftar tabel
\clearpage

\phantomsection
\listoffigures  % Daftar gambar
\clearpage

% \phantomsection
% \listofappendices
% \addcontentsline{toc}{chapter}{Daftar Lampiran}
% \clearpage


\enableboldchapterintoc

\end{singlespace}

%
% Gunakan penomeran Arab (1, 2, 3, ...) setelah bagian ini.
%
\pagenumbering{arabic} % Mengatur penomoran halaman ke format Arab
\fancyhf{} % Bersihkan header dan footer
\fancyfoot[R]{\thepage} % Nomor halaman di kanan bawah
%
%
%
%-----------------------------------------------------------------------------%
\chapter{\babSatu}
\thispagestyle{fancy}
%-----------------------------------------------------------------------------%
\todo{tambahkan kata-kata pengantar bab 1 disini jika perlu}

%-----------------------------------------------------------------------------%
\section{Latar Belakang}
%-----------------------------------------------------------------------------%
\todo{tuliskan latar belakang penelitian disini}


%-----------------------------------------------------------------------------%
\section{Rumusan Masalah}
%-----------------------------------------------------------------------------%
\todo{tuliskan rumusan masalah disini, bisa berbentuk point-point}


%-----------------------------------------------------------------------------%
\section{Tujuan Penelitian}
%-----------------------------------------------------------------------------%
\todo{tuliskan tujuan penelitian disini, diangkat dari rumusan masalah, bisa berbentuk point-point}

%-----------------------------------------------------------------------------%
\section{Manfaat Penelitian}
%-----------------------------------------------------------------------------%
\todo{Tuliskan tujuan penelitian.}

% ==============HAPUS DIBAWAH INI JIKA SUDAH TIDAK PERLU===================== %
% \subsection{Batasan Permasalahan}
% %-----------------------------------------------------------------------------%
% \todo{Umumnya ada asumsi atau batasan yang digunakan untuk 
% 	menjawab pertanyaan-pertanyaan penelitian diatas.}


% %-----------------------------------------------------------------------------%
% \section{Posisi Penelitian}
% %-----------------------------------------------------------------------------%
% \todo{Posisi penelitian Anda jika dilihat secara bersamaan dengan 
% 	peneliti-peneliti lainnya. Akan lebih baik lagi jika ikut menyertakan 
% 	diagram yang menjelaskan hubungan dan keterkaitan antar 
% 	penelitian-penelitian sebelumnya}


% %-----------------------------------------------------------------------------%
% \section{Metodologi Penelitian}
% %-----------------------------------------------------------------------------%
% \todo{Tuliskan metodologi penelitian yang digunakan.}


% %-----------------------------------------------------------------------------%
% \section{Sistematika Penulisan}
% %-----------------------------------------------------------------------------%
% Sistematika penulisan laporan adalah sebagai berikut:
% \begin{itemize}
% 	\item Bab 1 \babSatu \\
% 	\item Bab 2 \babDua \\
% 	\item Bab 3 \babTiga \\
% 	\item Bab 4 \babEmpat \\
% 	\item Bab 5 \babLima \\
% 	\item Bab 6 \babEnam \\
% 	\item Bab 7 \kesimpulan \\
% \end{itemize}

% \todo{Tambahkan penjelasan singkat mengenai isi masing-masing bab.}

 % Pendahuluan
%-----------------------------------------------------------------------------%
\chapter{\babDua}
\thispagestyle{fancy}
%-----------------------------------------------------------------------------%
\todo{tambahkan kata-kata pengantar bab 2 disini jika perlu}

% ==============HAPUS DIBAWAH INI JIKA SUDAH TIDAK PERLU===================== %
%-----------------------------------------------------------------------------%
\section{\latex~Secara Singkat}
%-----------------------------------------------------------------------------%
Definisi dari LaTeX \parencite{lankton2008introduction} adalah: \\ 
\begin{tabular}{| p{13cm} |}
	\hline 
	\\
	LaTeX is a family of programs designed to produce publication-quality 
	typeset documents. It is particularly strong when working with 
	mathematical symbols. \\	
	The history of LaTeX begins with a program called TEX. In 1978, a 
	computer scientist by the name of Donald Knuth grew frustrated with the 
	mistakes that his publishers made in typesetting his work. He decided 
	to create a typesetting program that everyone could easily use to 
	typeset documents, particularly those that include formulae, and made 
	it freely available. The result is TEX. \\	
	Knuth's product is an immensely powerful program, but one that does 
	focus very much on small details. A mathematician and computer 
	scientist by the name of Leslie Lamport wrote a variant of TEX called 
	LaTeX that focuses on document structure rather than such details. \\
	\\
	\hline
\end{tabular}

\vspace*{0.8cm}

Contoh sitasi lainnya menggunakan \verb|\parencite| adalah saat kita mau mensitasi pekerjaan tentang \textit{machine learning} \parencite{chin2000learning} dan \textit{dynamic programming} \parencite{barto1995learning}. \\

Dokumen \latex~sangat mudah, seperti halnya membuat dokumen teks biasa. Ada 
beberapa perintah yang diawali dengan tanda '\bslash'. 
Seperti perintah \bslash\bslash~yang digunakan untuk memberi baris baru. 
Perintah tersebut juga sama dengan perintah \bslash newline. 
Pada bagian ini akan sedikit dijelaskan cara manipulasi teks dan 
perintah-perintah \latex~yang mungkin akan sering digunakan. 
Jika ingin belajar hal-hal dasar mengenai \latex, silahkan kunjungi: 

\begin{itemize}
	\item \url{http://frodo.elon.edu/tutorial/tutorial/}, atau
	\item \url{http://www.maths.tcd.ie/~dwilkins/LaTeXPrimer/}
\end{itemize}


%-----------------------------------------------------------------------------%
\section{\latex~Kompiler dan IDE}
%-----------------------------------------------------------------------------%
Agar dapat menggunakan \latex~(pada konteks hanya sebagai pengguna), Anda 
tidak perlu banyak tahu mengenai hal-hal didalamnya. 
Seperti halnya pembuatan dokumen secara visual (contohnya Open Office (OO) 
Writer), Anda dapat menggunakan \latex~dengan cara yang sama. 
Orang-orang yang menggunakan \latex~relatif lebih teliti dan terstruktur 
mengenai cara penulisan yang dia gunakan, \latex~memaksa Anda untuk seperti 
itu.  

Kembali pada bahasan utama, untuk mencoba \latex~Anda cukup mendownload 
kompiler dan IDE. Saya menyarankan menggunakan Texlive dan Texmaker. 
Texlive dapat didownload dari \url{http://www.tug.org/texlive/}. 
Sedangkan Texmaker dapat didownload dari 
\url{http://www.xm1math.net/texmaker/}. 
Untuk pertama kali, coba buka berkas thesis.tex dalam template yang Anda miliki 
pada Texmaker. 
Dokumen ini adalah dokumen utama. 
Tekan F6 (PDFLaTeX) dan Texmaker akan mengkompilasi berkas tersebut menjadi 
berkas PDF. 
Jika tidak bisa, pastikan Anda sudah menginstall Texlive. 
Buka berkas tersebut dengan menekan F7. 
Hasilnya adalah sebuah dokumen yang sama seperti dokumen yang Anda baca saat 
ini. 


%-----------------------------------------------------------------------------%
\section{Bold, Italic, dan Underline}
%-----------------------------------------------------------------------------%
Hal pertama yang mungkin ditanyakan adalah bagaimana membuat huruf tercetak 
tebal, miring, atau memiliki garis bawah. 
Pada Texmaker, Anda bisa melakukan hal ini seperti halnya saat mengubah dokumen 
dengan OO Writer. 
Namun jika tetap masih tertarik dengan cara lain, ini dia: 

\begin{itemize}
	\item \bo{Bold} \\
		Gunakan perintah \bslash textbf$\lbrace\rbrace$ atau 
		\bslash bo$\lbrace\rbrace$. 
	\item \f{Italic} \\
		Gunakan perintah \bslash textit$\lbrace\rbrace$ atau 
		\bslash f$\lbrace\rbrace$. 
	\item \underline{Underline} \\
		Gunakan perintah \bslash underline$\lbrace\rbrace$.
	\item $\overline{Overline}$ \\
		Gunakan perintah \bslash overline. 
	\item $^{superscript}$ \\
		Gunakan perintah \bslash $\lbrace\rbrace$. 
	\item $_{subscript}$ \\
		Gunakan perintah \bslash \_$\lbrace\rbrace$. 
\end{itemize}

Perintah \bslash f dan \bslash bo hanya dapat digunakan jika package 
uithesis digunakan. 


%-----------------------------------------------------------------------------%
\section{Memasukan Gambar}
%-----------------------------------------------------------------------------%
Setiap gambar dapat diberikan caption dan diberikan label. Label dapat 
digunakan untuk menunjuk gambar tertentu. 
Jika posisi gambar berubah, maka nomor gambar juga akan diubah secara 
otomatis. 
Begitu juga dengan seluruh referensi yang menunjuk pada gambar tersebut. 
Contoh sederhana adalah \pic~\ref{fig:testGambar}. 
Silahkan lihat code \latex~dengan nama bab2.tex untuk melihat kode lengkapnya. 
Harap diingat bahwa caption untuk gambar selalu terletak dibawah gambar. 

\begin{figure}
	\centering
	\includegraphics[width=0.50\textwidth]
		{assets/pics/creative_common.png}
	\caption{\license.}
	\label{fig:testGambar}
\end{figure}


%-----------------------------------------------------------------------------%
\section{Membuat Tabel}
%-----------------------------------------------------------------------------%
Seperti pada gambar, tabel juga dapat diberi label dan caption. 
Caption pada tabel terletak pada bagian atas tabel. 
Contoh tabel sederhana dapat dilihat pada \tab~\ref{tab:tab1}.

\begin{table}
	\centering
	\caption{Contoh Tabel}
	\label{tab:tab1}
	\begin{tabular}{| l | c r |}
		\hline
		& kol 1 & kol 2 \\ 
		\hline
		baris 1 & 1 & 2 \\
		baris 2 & 3 & 4 \\
		baris 3 & 5 & 6 \\
		jumlah  & 9 & 12 \\
		\hline
	\end{tabular}
\end{table}

Ada jenis tabel lain yang dapat dibuat dengan \latex~berikut 
beberapa diantaranya. 
Contoh-contoh ini bersumber dari 
\url{http://en.wikibooks.org/wiki/LaTeX/Tables}

\begin{table}
	\centering
	\caption{An Example of Rows Spanning Multiple Columns}
	\label{row.spanning}
	\begin{tabular}{|l|l|*{6}{c|}}
  		\hline % create horizontal line
  		No & Name & \multicolumn{3}{|c|}{Week 1} & \multicolumn{3}{|c|}{Week 2} \\
  		\cline{3-8} % create line from 3rd column till 8th column
  		& & A & B & C & A & B & C\\
  		\hline
  		1 & Lala & 1 & 2 & 3 & 4 & 5 & 6\\
  		2 & Lili & 1 & 2 & 3 & 4 & 5 & 6\\
  		3 & Lulu & 1 & 2 & 3 & 4 & 5 & 6\\
  		\hline
	\end{tabular}
\end{table}

\begin{table}
	\centering
	\caption{An Example of Columns Spanning Multiple Rows}
	\label{column.spanning}
	\begin{tabular}{|l|c|l|}
		\hline
		Percobaan & Iterasi & Waktu \\
		\hline
		Pertama & 1 & 0.1 sec \\ \hline
		\multirow{2}{*}{Kedua} & 1 & 0.1 sec \\
 		& 3 & 0.15 sec \\ 
 		\hline
		\multirow{3}{*}{Ketiga} & 1 & 0.09 sec \\
 		& 2 & 0.16 sec \\
 		& 3 & 0.21 sec \\ 
 		\hline
	\end{tabular}
\end{table}

\begin{table}
	\centering
	\caption{An Example of Spanning in Both Directions Simultaneously}
	\label{mix.spanning}
	\begin{tabular}{cc|c|c|c|c|}
		\cline{3-6}
		& & \multicolumn{4}{|c|}{Title} \\ \cline{3-6}
		& & A & B & C & D \\ \hline
		\multicolumn{1}{|c|}{\multirow{2}{*}{Type}} &
		\multicolumn{1}{|c|}{X} & 1 & 2 & 3 & 4\\ \cline{2-6}
		\multicolumn{1}{|c|}{}                        &
		\multicolumn{1}{|c|}{Y} & 0.5 & 1.0 & 1.5 & 2.0\\ \cline{1-6}
		\multicolumn{1}{|c|}{\multirow{2}{*}{Resource}} &
		\multicolumn{1}{|c|}{I} & 10 & 20 & 30 & 40\\ \cline{2-6}
		\multicolumn{1}{|c|}{}                        &
		\multicolumn{1}{|c|}{J} & 5 & 10 & 15 & 20\\ \cline{1-6}
	\end{tabular}
\end{table}

 % Tinjauan Kepustakaan
%-----------------------------------------------------------------------------%
\chapter{\babTiga}
\thispagestyle{fancy}
%-----------------------------------------------------------------------------%
\todo{tambahkan kata-kata pengantar bab 1 disini}

%-----------------------------------------------------------------------------%
\section{Waktu dan Lokasi Penelitian}
%-----------------------------------------------------------------------------%
\todo{tuliskan waktu dan lokasi penelitian disini}

%-----------------------------------------------------------------------------%
\section{Alat dan Bahan}
%-----------------------------------------------------------------------------%
\todo{tuliskan alat dan bahan penelitian disini}

%-----------------------------------------------------------------------------%
\section{Metode Penelitian}
%-----------------------------------------------------------------------------%
\todo{tuliskan metode penelitian disini}
 % Metodologi Penelitian

% Agar nomor halaman daftar pustaka halaman pertama berada di kanan bawah karena \printbibliography menggunakan plain
\fancypagestyle{plain}{ %This is to change the \chapter functionality so the page number is in the bottom right corner.
  \fancyhf{}
  \fancyfoot[R]{\thepage}  % Nomor halaman di kanan bawah
  \renewcommand{\headrulewidth}{0pt}
  \renewcommand{\footrulewidth}{0pt}
}
% Alternatif manajemen daftar pustaka dengan \bibliography
\begin{singlespace}
    \clearpage
    \addChapter{\bo{DAFTAR KEPUSTAKAAN}}
    \printbibliography[title={Daftar Kepustakaan}]
\end{singlespace}


%
% Lampiran 
%
% \begin{appendix}
% 	% \input{_internals/markLampiran} % Penggunaan ini hanya untuk menandai halaman LAMPIRAN, jika perlu
% 	% \setcounter{page}{2}\textbf{} % Reset penomoran halaman jika perlu
% 	\input{src/99-back_matter/lampiran}
% \end{appendix}

\end{document}