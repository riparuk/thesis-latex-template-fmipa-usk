%-----------------------------------------------------------------------------%
\chapter{\babSatu}
\thispagestyle{fancy}
%-----------------------------------------------------------------------------%
\todo{tambahkan kata-kata pengantar bab 1 disini jika perlu}

%-----------------------------------------------------------------------------%
\section{Latar Belakang}
%-----------------------------------------------------------------------------%
\todo{tuliskan latar belakang penelitian disini}


%-----------------------------------------------------------------------------%
\section{Rumusan Masalah}
%-----------------------------------------------------------------------------%
\todo{tuliskan rumusan masalah disini, bisa berbentuk point-point}
\begin{enumerate}
	\item ...
	\item ...
	\item ...
\end{enumerate}


%-----------------------------------------------------------------------------%
\section{Tujuan Penelitian}
%-----------------------------------------------------------------------------%
\todo{tuliskan tujuan penelitian disini, diangkat dari rumusan masalah, bisa berbentuk point-point}
\begin{enumerate}
	\item ...
	\item ...
	\item ...
\end{enumerate}
%-----------------------------------------------------------------------------%
\section{Manfaat Penelitian}
%-----------------------------------------------------------------------------%
\todo{Tuliskan tujuan penelitian.}

% ==============HAPUS DIBAWAH INI JIKA SUDAH TIDAK PERLU===================== %
% \subsection{Batasan Permasalahan}
% %-----------------------------------------------------------------------------%
% \todo{Umumnya ada asumsi atau batasan yang digunakan untuk 
% 	menjawab pertanyaan-pertanyaan penelitian diatas.}


% %-----------------------------------------------------------------------------%
% \section{Posisi Penelitian}
% %-----------------------------------------------------------------------------%
% \todo{Posisi penelitian Anda jika dilihat secara bersamaan dengan 
% 	peneliti-peneliti lainnya. Akan lebih baik lagi jika ikut menyertakan 
% 	diagram yang menjelaskan hubungan dan keterkaitan antar 
% 	penelitian-penelitian sebelumnya}


% %-----------------------------------------------------------------------------%
% \section{Metodologi Penelitian}
% %-----------------------------------------------------------------------------%
% \todo{Tuliskan metodologi penelitian yang digunakan.}


% %-----------------------------------------------------------------------------%
% \section{Sistematika Penulisan}
% %-----------------------------------------------------------------------------%
% Sistematika penulisan laporan adalah sebagai berikut:
% \begin{itemize}
% 	\item Bab 1 \babSatu \\
% 	\item Bab 2 \babDua \\
% 	\item Bab 3 \babTiga \\
% 	\item Bab 4 \babEmpat \\
% 	\item Bab 5 \babLima \\
% 	\item Bab 6 \babEnam \\
% 	\item Bab 7 \kesimpulan \\
% \end{itemize}

% \todo{Tambahkan penjelasan singkat mengenai isi masing-masing bab.}

